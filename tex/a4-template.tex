\documentclass[version=last,fontsize=\fontsizevar]{scrartcl}
\usepackage[
    a4paper,
    total={200mm,281mm},
    left=9mm,
    right=4mm,
    top=5mm,
    bottom=5mm
  ]{geometry}
\usepackage[english,french,russian]{babel}
\usepackage{tabularx}
\usepackage{longtable}
\usepackage{multicol}
\usepackage{fontspec}
\usepackage{csvsimple}
\usepackage{url}
\setmainfont{XCharter}
\usepackage[table]{xcolor}
\usepackage{ifthen}
\pagestyle{empty}
\usepackage{marginnote}
\usepackage{setspace}
\usepackage{forloop}
\usepackage{intcalc}
\usepackage{ifthen}

\usepackage{scrlayer-scrpage}
\usepackage{scrlayer-notecolumn}

\newsavebox\ltmcbox

\setlength{\columnsep}{0mm} % adjust spacing between columns
\setlength{\columnseprule}{0.1pt} % width of drawn line between columns
\setlength\tabcolsep{0.4mm} % spacing in table: between Russian verbs and translation

\renewcommand{\baselinestretch}{\baselinevar}

\setkomafont{notecolumn.marginpar}{\KOMAoptions{fontsize=\fontsizevar}\setstretch{\baselinevar}\fontfamily{XCharter}\selectfont\raggedleft}

\newcommand{\marginlineno}{
  % Calc num lines to set numbers for
  \def\i{616}
  \def\j{\numcolumns}
  % The next thing we have to do is convert \x and \y 
  % (which TeX views as merely the characters 3 and 0, 1 and 0, 
  % not as the decimal numbers 30 and 10) into count values
  \newcount\a\a=\number\i
  \newcount\b\b=\number\j
  \divide\a by\b
  \def\i{\the\a}

  \reversemarginpar
  \setlength\marginparwidth{4mm}
  \setlength\marginparsep{1.5mm}
  \makenote*{
    \noindent 1\\
    \newcounter{x}
    \forloop{x}{2}{\value{x}<176}{
      \ifthenelse{\equal{\intcalcMod{\value{x}}{5}}{0}}{
        \noindent \arabic{x}\\
      }{
        \vspace{\baselineskip}
      }
    }
  }
}

\begin{document}
  \marginlineno

  \begin{multicols}{\numcolumns}
  \setbox\ltmcbox\vbox{
  \makeatletter\col@number\@ne

  \ifthenelse{\equal{\withyellow}{yes}}{
    \csvstyle{myStyle}{
      longtable=p{\widthleftcol}>{\small}p{\widthrightcol},
      separator=semicolon,
      % Dirty fix to set colored rows. Doing a different way brought linebreaks for every colored row.
      % Bug: 1st row doesn't get colored even when it should be
      late after line=\csvifoddrow{\ifcsvstrcmp{\yellow}{1}{\\\rowcolor{yellow}}{\\}}{\ifcsvstrcmp{\yellow}{1}{\\\rowcolor{yellow}}{\\}}
    }
  }{
    \csvstyle{myStyle}{
      longtable=p{\widthleftcol}>{\small}p{\widthrightcol},
      separator=semicolon
    }
  }

  \csvreader[myStyle]{\csvfilename}
    {verb=\verb,\transfield=\trans,yellow=\yellow}
    {\verb & \trans} % specify your columns here
  \unskip
  \unpenalty
  \unpenalty
  }
  \unvbox\ltmcbox

  \begin{addmargin}[1mm]{0mm}
    \input{\footerfile}
  \end{addmargin}

  \end{multicols}
\end{document}
